O problema de ordenação de sequências numéricas surge frequentemente em aplicações computacionais. Este problema pode ser definido formalmente da seguinte maneira \cite{cormen2009introduction}:
\begin{itemize}
    \item \textbf{Entrada:} uma sequência de $n$ números, $a_1, a_2, ..., a_n$
    \item \textbf{Saída:} uma permutação (reordenação) da sequência de entrada, $a'_1, a'_2,...,a'_n$ tal que $a'_1 \leq a'_2 \leq ... \leq a'_n$ 
\end{itemize}

Existem diversos algoritmos de ordenação que resolvem o problema apresentado acima, \textit{e.g.} \textit{insertion sort}, \textit{selection sort}, \textit{merge sort}, \textit{quicksort}, dentre outros. Neste trabalho, vamos analisar um dos algoritmos de ordenação mais simples, o \textit{bubblesort}. A ideia geral do \textit{bubblesort} é percorrer diversas vezes o vetor de entrada e, a cada passagem, mover para o final da porção ainda não ordenada o maior elemento. Uma implementação recursiva desse algoritmo é apresentada a seguir (Algoritmo \ref{algo:bubblesort}).

\begin{algorithm}[!h]
    \begin{algorithmic}[1]
        \Function{bubblesort}{int array $A$, int $n$}
            \If{n = 1}
                \State \Return
            \EndIf
            \For{$i \leftarrow 0$ to $n-1$}
                \If{$A[i] > A[i+1]$}
                    \State swap($A[i], A[i+1]$)
                \EndIf
            \EndFor
            \State bubblesort($A$,$n-1$)
            
        
        \EndFunction
    \end{algorithmic}\vspace{-2pt}
    \caption{Implementação recursiva do \textit{bubblesort}}
    \label{algo:bubblesort}
\end{algorithm}

Uma das formas de comparar o desempenho do \textit{bubblesort} com o desempenho de outros algoritmos de ordenação é através de uma análise de complexidade.
A complexidade do \textit{bubblesort} é de ordem quadrática ($O(n^2)$), \textit{i.e.}, no pior caso, são feitas $n^2$ trocas durante a ordenação, onde $n$ é o número de elementos do vetor de entrada.
Neste trabalho, um assistente de prova será utilizado para provar a complexidade do \textit{bubblesort}. 
A prova será realizada utilizando o \textit{Prototype Verification System} (PVS) \cite{owre1992pvs}. \\

Os principais objetivos deste trabalho são:
\begin{itemize}
    \item Implementar uma versão recursiva do \textit{bubblesort} com um contador para o número de trocas realizadas durante a ordenação;
    \item Provar a complexidade assintótica do \textit{bubblesort} utilizando o PVS.
\end{itemize}

%Na próxima seção, será realizada uma apresentação detalhada do problema.