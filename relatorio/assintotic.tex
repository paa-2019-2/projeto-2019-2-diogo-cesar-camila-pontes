\subsubsection{Funções auxiliares}

Para a realização da análise assintótica do Bubblesort,
inicialmente foram definidas três funções auxiliares
com o objetivo de rastrear as contagens do número total
de comparações realizadas pelo algoritmo após retornar
a lista ordenada.

\begin{itemize}
\item \texttt{bubbling\_count}: recebe uma lista $l$, um
natural $n$ (equivalentes aos parâmetros da função
\texttt{bubbling} original) e um contador $count$, que é
incrementando quando alguma comparação for realizada.
Seu valor de retorno é o par $(l, count)$ com a lista e
o contador devidamente atualizados.
\item \texttt{bubblesort\_aux\_count}: recebe uma lista $l$,
um natural $n$ (equivalentes aos parâmetros da função
\texttt{bubblesort\_aux} original) e um contador $count$,
que é passado para a função \texttt{bubbling\_count}
chamada internamente. Seu valor de retorno é o par
$(l, count)$ com a lista e o contador devidamente
atualizados.
\item \texttt{bubblesort\_count}: recebe uma lista $l$,
um natural $n$, equivalentes aos parâmetros da função
\texttt{bubblesort} original. Ela chama
\texttt{bubblesort\_aux\_count} da mesma forma que
\texttt{bubblesort} chama \texttt{bubblesort\_aux},
mas com o parâmetro do contador iniciando em 0. Seu valor
de retorno é o par $(l, count)$ com a lista ordenada
e o contador atualizados com o número total de comparações
realizadas pelo algoritmo.
\end{itemize}

\subsubsection{Lemas}
Lemas utilizados na prova da complexidade de \texttt{bubbling\_count}:
\begin{itemize}
    \item \texttt{bubbling\_equiv}: atesta a equivalência entre \texttt{bubbling} e \texttt{bubbling\_count}
    \item \texttt{bubbling\_length}: afirma que a função \texttt{bubbling\_count} não altera o tamanho da lista de entrada
    \item \texttt{bubbling\_counts\_n}: afirma que \texttt{bubbling\_count} realiza exatamente $n$ comparações, onde $n$ é o tamanho da lista de entrada, e que, portanto, sua complexidade é linear
\end{itemize}
Lemas utilizados na prova da complexidade de \texttt{bubblesort\_aux\_count}:
\begin{itemize}
    \item \texttt{bubblesort\_aux\_equiv}: atesta a equivalência entre \texttt{bubblesort\_aux} e \texttt{bubblesort\_aux\_count}
    \item \texttt{bubblesort\_counts\_n2}: afirma que \texttt{bubblesort\_aux\_count} realiza exatamente $n(n+1)/2$ comparações, onde $n$ é o tamanho da lista de entrada, e que, portanto, sua complexidade é quadrática
\end{itemize}
Lemas utilizados na prova da complexidade de \texttt{bubblesort\_count}:
\begin{itemize}
    \item \texttt{bubblesort\_equiv}: atesta a equivalência entre \texttt{bubblesort} e \texttt{bubblesort\_count}
    \item \texttt{bubblesort\_counts\_n2}: afirma que \texttt{bubblesort\_count} realiza exatamente $n(n-1)/2$ comparações, onde $n$ é o tamanho da lista de entrada, e que, portanto, sua complexidade é linear
\end{itemize}


