
O problema em questão consiste na análise da complexidade
assintótica do algoritmo Bubblesort. A resolução proposta
foi decomposta em três partes: na elaboração de funções
auxiliares que possibilitem a realização da análise;
na elaboração e prova de lemas que dizem respeito a análise
da complexidade do algoritmo; e por fim, na elaboração e prova
de lemas que garantam a equivalência entre as funções auxiliares
e as funções originalmente implementadas.

\subsection{Funções auxiliares}

Para a realização da análise assintótica do Bubblesort,
inicialmente foram definidas três funções auxiliares
com o objetivo de rastrear as contagens do número total
de comparações realizadas pelo algoritmo após retornar
a lista ordenada.

\begin{itemize}
\item \texttt{bubbling\_count}: recebe uma lista $l$, um
natural $n$ (equivalentes aos parâmetros da função
\texttt{bubbling} original) e um contador $count$, que é
incrementando quando alguma comparação for realizada.
Seu valor de retorno é o par $(l, count)$ com a lista e
o contador devidamente atualizados.
\item \texttt{bubblesort\_aux\_count}: recebe uma lista $l$,
um natural $n$ (equivalentes aos parâmetros da função
\texttt{bubblesort\_aux} original) e um contador $count$,
que é passado para a função \texttt{bubbling\_count}
chamada internamente. Seu valor de retorno é o par
$(l, count)$ com a lista e o contador devidamente
atualizados.
\item \texttt{bubblesort\_count}: recebe uma lista $l$,
um natural $n$, equivalentes aos parâmetros da função
\texttt{bubblesort} original. Ela chama
\texttt{bubblesort\_aux\_count} da mesma forma que
\texttt{bubblesort} chama \texttt{bubblesort\_aux},
mas com o parâmetro do contador iniciando em 0. Seu valor
de retorno é o par $(l, count)$ com a lista ordenada
e o contador atualizados com o número total de comparações
realizadas pelo algoritmo.
\end{itemize}

\subsection{Lemas}

Como o objetivo é a análise da complexidade assintótica do algoritmo Bubblesort,
dividimos a análise com base nas funções que compõem a implementação. Para cada
uma das unções auxiliares elaboramos ao menos dois lemas: um para analisar seu
comportamento assintótico e um segundo para garantir sua equivalência com a
função correspondente original.

\hspace{0.27cm} Lemas utilizados na prova da complexidade de \texttt{bubbling\_count}:
\begin{enumerate}
    \item \label{lemma:bcount} \texttt{bubbling\_counts\_n}: afirma que \texttt{bubbling\_count} realiza exatamente $n$ comparações, onde $n$ é o tamanho da lista de entrada, e que, portanto, sua complexidade é linear
    \item \label{lemma:bequiv} \texttt{bubbling\_equiv}: afirma que \texttt{bubbling} e \texttt{bubbling\_count} são equivalentes
    \item \label{lemma:blen} \texttt{bubbling\_length}: afirma que a função \texttt{bubbling\_count} não altera o tamanho da lista de entrada

\vspace{1cm}
Lemas utilizados na prova da complexidade de \texttt{bubblesort\_aux\_count}:

    \item \label{lemma:bacount} \texttt{bubaux\_counts\_n2}: afirma que \texttt{bubblesort\_aux\_count} realiza exatamente $n(n+1)/2$ comparações, onde $n$ é o tamanho da lista de entrada, e que, portanto, sua complexidade é quadrática
    \item \label{lemma:baequiv} \texttt{bubblesort\_aux\_equiv}: afirma que \texttt{bubblesort\_aux} e \texttt{bubblesort\_aux\_count} são equivalentes

\vspace{1cm}
Lemas utilizados na prova da complexidade de \texttt{bubblesort\_count}:

    \item \label{lemma:bscount} \texttt{bubblesort\_counts\_n2}: afirma que \texttt{bubblesort\_count} realiza exatamente $n(n-1)/2$ comparações, onde $n$ é o tamanho da lista de entrada, e que, portanto, sua complexidade é quadrática
    \item \label{lemma:bsequv} \texttt{bubblesort\_equiv}: afirma que \texttt{bubblesort} e \texttt{bubblesort\_count} são equivalentes
\end{enumerate}

\subsection{Análise assintótica}

As provas podem ser verificadas por completo através do PVS a partir
dos arquivos que acompanham este projeto. O arquivo \texttt{bubblesort2.pvs}
contem a implementação das funções originais fornecidas pelo professor da
disciplina, bem como a implementação das funções auxiliares e dos lemas
elaborados pelo grupo. Apresentaremos a seguir um detalhamento dos pontos que
consideramos fundamentais na realização das provas. Adotaremos a seguinte notação
para as fórmulas nesta Seção: se $l$ é uma lista, $|l|$ indica o seu comprimento.
O $i$-ésimo elemento de $l$ é dado por $l_i$ e uma lista $l_i:l'_i$ possui o
elemento $l_i$ na cabeça e a lista $l'_i$ na cauda.
Analogamente, $P_i$ denota o $i$-ésimo elemento de um par ordenado $P=(a,b)$ e
$f(x)_i$ o $i$-ésimo elemento de uma função $f(x)=(a,b)$ que retorna um par.

